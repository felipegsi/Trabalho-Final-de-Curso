%!TEX root = ../template.tex
%%%%%%%%%%%%%%%%%%%%%%%%%%%%%%%%%%%%%%%%%%%%%%%%%%%%%%%%%%%%%%%%%%%
%% chapter1.tex
%% NOVA thesis document file
%%
%% Chapter with introduction
%%%%%%%%%%%%%%%%%%%%%%%%%%%%%%%%%%%%%%%%%%%%%%%%%%%%%%%%%%%%%%%%%%%

\typeout{NT FILE chapter1.tex}%

\chapter{Introdução}
\label{cha:introduction}

\section{Teste}

\subsection{Teste2}
%\prependtographicspath{{Chapters/Figures/Covers/}}

% epigraph configuration
%\epigraphfontsize{\small\itshape}
%\setlength\epigraphwidth{12.5cm}
%\setlength\epigraphrule{0pt}

\begin{figure}[htbp]
	\centering
	\includegraphics[width=0.1\linewidth]{NOVAthesisFiles/Images/novathesis-insignia}
	\label{fig:1}
	\caption{league of legends}
\end{figure}

\includegraphics[width=0.875\linewidth]{NOVAthesisFiles/Images/novathesis-text}


REF->(\cite{novathesis-manual}) Este é uma maneira de citar

\begin{figure}[htbp]
	\centering
	\subfloat[Model's Section~\ref{cha:introduction} trainning RMSE and Loss results.]{%
		\includegraphics[width=.1\linewidth]{NOVAthesisFiles/Images/novathesis-insignia}}
	\hfil
	\subfloat[Model's Section~\ref{cha:introduction} trainning RMSE and Loss results.]{%
		\includegraphics[width=.1\linewidth]{NOVAthesisFiles/Images/novathesis-insignia}}
	\caption{RMSE and loss training results for trajectory model and phase-shift model with the static and dynamic group movement.}
	\label{fig:models_results_lstms_loss} 
\end{figure}